\documentclass[11pt, a4paper]{article}

\usepackage{amsmath}
\usepackage{amsfonts} %Matheschriften
\usepackage{amssymb} %Mathesymbole
%\usepackage{mathptmx} % Einstellung für Schriften und Sonderzeichen in mathematischen Umgebungen
                        % ändert SChriftfont
\usepackage{wasysym} % Stellt diverse Sonderzeichen bereit
\usepackage{siunitx}
\usepackage{float}
\usepackage{microtype}
\usepackage{graphicx}
\usepackage{hyperref}
\usepackage{xcolor}
\usepackage[section]{placeins}


\usepackage[ngerman]{babel}
\addto\captionsngerman{%
 \renewcommand{\abstractname}{Abstract}}

\title{Versuch 2: Pendel}
\author{Jascha Fricker, Benedict Brouwer}

\begin{document}
    \maketitle

    

    \begin{abstract}
        In diesen beiden Versuchsaufbauten werden verschiedene Pendel untersucht. Zum einen wird mit
        einem Reversionspendel die Erdbeschleunigung $g$ gemessen, zum anderen werden zwei mit einer Feder
        gekoppelte Pendel untersucht und mit den Messwerten u.a. die Federkonstante berechnet. 
    \end{abstract}

    \tableofcontents

    \newpage

    \section{Reversionspendel}

    \subsection{Experimenteller Aufbau und Theorie}

    Ein Reversionspendel hat zwei Aufhängepunkte und zwei Massen, die alle auf einer Geraden liegen.
    Dabei kann eine Masse verschoben werden.
    Es gibt zwei Positionen des verschiebbaren Gewichts, an dem die Periode der Schwingung an beiden
    Aufhängepunkten gleich ist. Für ein Reversionspendel in dieser Konfiguration ist die Periode
    \begin{align}
         \tau^2 &= 4\pi^2 \cdot \frac{l_r}{g} \\
        \Rightarrow \ \ g &= 4\pi^2 \cdot \frac{l_r}{\tau^2} \label{gformel}
    \end{align}
    (Herleitung siehe \cite[Abschitt 1.3]{pen}). Daraus ist ersichtlich, dass, wenn dieser Fall eintritt, die Periodendauer unabängig von der Masse und des
    Trägheitsmoments ist. So kann mit der Periodendauer und dem Abstand der beiden Aufhängungspunkte $l_r$
    die Erdbeschleunigung ausgerechnet werden.
    Um diese besonderen Positionen der zweite Masse zu finden, wird im Experiment die Periodendauer
    mit der Masse an verschieden Positionen von beiden Aufhängungen gemessen,
    um dann mithilfe des Schnittpunkts der Ausgleichgeraden die gewünschten
    Punkte  zu bestimmen.



    \subsection{Ergebnisse}

    \begin{figure}[!h]
        \centering
        \includegraphics[width=120mm]{./Reversion_grob.pdf}

        \caption{Grobe Darstellung der Messdaten (der Abstand ist nur relativ)}
        \label{fig:revgrob}
    \end{figure}
    \begin{figure}[!h]
        \centering
        \includegraphics[width=\textwidth]{./Reversion_fein.pdf}

        \caption{Genauere Darstellung der Schnittpunkte}
        \label{fig:revfein1}
    \end{figure}

    Im Graphen \ref{fig:revgrob} wurde die Periodendauer abhängig vom Ort der zweiten Masse für beide Aufhängungspunkte dargestellt. 
    In dem Graphen \ref{fig:revfein1}  können die beiden Schnittpunkte der Periodendauer der verschiedenen
    Aufhängungen nochmal näher gesehen werden. Die Daten wurden mit einem Polynom 6. Grades gefittet.
    Leider ist dies aber noch nicht genau genug, sodass einige Punkte ausserhalb
    der Konfidenzintervalls liegen. Dies ist uns aber erst nachher aufgefallen. Für den Fehler der Zeitmessung
    wurden $2 \si{\milli\second}$ angenommen, da in den Messdaten eine maximale Abweichung von $1\si{\milli\second}$ bei verschiedenen Messungen 
    der gleichen Periodendauer vorkam und die Lichtchranke selber auch eine Genauigkeit von $1\si{\milli\second}$ hat.
    Beachtenswert ist, dass das Pendel einge Zeit braucht bis es sich "eingependelt" hat und konsistente Messwerte
    gemessen werden können. Der Abstand der beiden Aufhängungspunkte $l_r$ beträgt $800,0(11)\si{\centi\metre}$.
    In Tabelle \ref{ergrev} sind die Messdaten und berechneten Werte dargestellt. 

    \begin{table}[H]
        \centering
        \begin{tabular}{c c c}
            & Schnittpunkt 1 & Schnittpunkt 2 \\ \hline
            Periodendauer &  $1794,96(456)\si{\milli\second} $ & $1795,00(234)\si{\milli\second} $ \\
            Erdbeschleunigung & $9,803(51)\si{\metre\per\second\squared}$ & $9,802(26)\si{\metre\per\second\squared}$ \\
            gewichtetes Mittel & \multicolumn{2}{c}{$g = 9,802(23)\si{\metre\per\square\second}$}
        \end{tabular}
        \caption{Ergebnisse Reversionspendel}
        \label{ergrev}
    \end{table}

    \subsection{Diskussion}
    Beide berechneten Periodendauern und Erdbeschleunigungen liegen sehr nahe bei einander.
    Der in München vom International Gravimetric Bureau gemesse Wert von $g = 9,807232\si{\metre\per\second\squared}$ \cite{glit} liegt
    eindeutig im Konfidenzintervall unserer Messungen und sogar relativ nahe am berechneten Wert.


    \section{Gekoppelte Pendel}

    \subsection{Experimenteller Aufbau und Theorie}

    In diesem Experiment wurden zwei Pendel beobachtet, die mit einer Feder auf einer einstellbaren Höhe
    gekoppelt wurden. Solche Pendel haben zwei Fundamentalschwingungen. 
    Die gleichphasige Winkelgeschwindigkeit 
    \begin{align}
        \omega_{gl} = \sqrt{\frac{mgl}{J}}
    \end{align}
    
    der Schwingung
    wird nur durch das Trägheitsmoments $J$, die Masse $m$ und dem Abstand $l$ des Schwerpunkts zur Aufhängung bestimmt.
    Die gegenphasige Schwingung mit Winkelgeschwindigkeit. 
    \begin{align}
        \omega_{geg} = \sqrt{\frac{mgl+2\kappa r^2}{J}}
    \end{align}
    beeinflusst hingegegen auch die Feder mit Federkonstante $\kappa$ und Befestigunsabstand $r$. \\
    In diesem Versuch wurden drei Messreihen aufgenommen: Eine gleichphasige, eine gegenphasige und
    eine mit Schwebung.\\
    Für die Ausarbeitung müssen verschiedene charakterischtische Werte des Systems \textit{gekoppelte Pendel} bestimmt werden. \\

    Den Koplungsgrad $K$

    \begin{align} \label{Kformel}
        K = \frac{\omega_{geg}^2 - \omega_{geg}^2}{\omega_{geg}^2 + \omega_{geg}^2} \ \ \ \text{Siehe: \cite[Formel (35)]{pen}}
    \end{align}
     bestimmt man mithilfe der beide Winkelgeschwindigkeiten der gleichphasigen und gegenphasigen Schwingung. \\

    Aus der Schwebung können mithilfe der Schwebungskreisfrequenz $\omega_S$
    und der mitteleren Kreisfrequenz  $\omega_M$ zuerst die Grundschwingungsfrequenzen
    \begin{align}
        \omega_{geg} &= \frac{\omega_{geg} + \omega_{gl}}{2} + \frac{\omega_{geg} - \omega_{gl}}{2} = \omega_M + \omega_S \\
        \omega_{gl} &= \frac{\omega_{geg} + \omega_{gl}}{2} - \frac{\omega_{geg} - \omega_{gl}}{2} = \omega_M - \omega_S
    \end{align}
    ausgerechnet werden (Siehe: \cite[Formel (34)]{pen}). Um dann wieder den Kopplungsgrad nach Formel (\ref{Kformel})
    zu bestimmen. \\

    Mit den gegeben (Theorie-) Daten in \textbf{Aufgabe 13} \cite{pen} (Masse Pendelgewicht $M_P = 1,080(20)\si{\kilogram}$,
    Masse Stange $M_S = 0,160(5)\si{\kilogram}$, Abstand Gewicht-Aufhängung $l = 0,968(10)\si{\metre}$ und Länge $l_0 0 1,030(20)\si{\metre}$
    und den Formeln der Modellbildung \cite{modell}
    lassen sich Trägheitsmoment
    \begin{align}
        J = 1,202(52)\si{\kilogram\metre\squared} \label{J}
    \end{align}
    und Drehmomentkonstante
    \begin{align}
        D = 11,06(30)\si{\newton\metre} \label{D}
    \end{align} 
    ausrechnen (Fehlerberechnung siehe \cite[Formel (19)]{ABW}) und damit die Modellkurven
    mithlife der Formeln \cite[(30)]{pen} und \cite[(35)]{pen}
    aufstellen
    
    \begin{align}
        K &= \frac{\kappa r^2}{D + \kappa r^2} \label{Kkappa} \\
        \omega_{geg} &= \sqrt{\frac{D+2\kappa r^2}{J}} \label{Okappa}
    \end{align}
    um letzendlich durch einen Fit  $\kappa$ zu bestimmen

    \subsection{Ergebnisse und Diskussion}

    
    \begin{figure}[!h]
        \centering
        \includegraphics[width=\textwidth]{./plotW.pdf}

        \caption{Kreisfrequenzen der beiden Fundamentalschwingungen}
        \label{fig:plotW}
    \end{figure}
    \begin{figure}[!h]
        \centering
        \includegraphics[width=\textwidth]{./Kplot.pdf}

        \caption{Kopplungsgrad der beiden Federn}
        \label{fig:plotK}
    \end{figure}

    Die verschieden Kreisfrequenzen wurden bestimmt, indem die jeweilligen Theoriekurven mit geeigneten
    Bounds an die Daten gefittet wurden, sodass sowohl die Frequenz als auch die Unsicherheit ausgegeben
    werden konnte. Die gesammten Messwerte sind im Anhang \ref{kmess} zu finden.


    In Graphen \ref{fig:plotW} sind die anhand der zwei verschieden Messreihen
    berechneten $\omega_{geg}$ und $\omega_{gl}$ dargestellt. Die Ungenauigkeiten der 1. und 2. Messreihe (Aufgabe 5) sind so klein,
    dass sie nicht aufgetragen werden konnten. Bei den anhand der Schwebung berechneten Frequenzen (Messreihe 3)
    sind zwar die Ungenauigkeiten größer, aber die Punkte aus der 1. und 2. Messreihe liegen im Konfidenzintervall.
    Die Fehler wurden durch die Genauigkeit der gefitteten Funktionen bestimmt. Systematische Fehler wurden
    nicht berücksicht, da diese nur minimal sind. Auch die Fehler der Längenmessung von $r$ wurden vernachlässigt.
    Der Theoriewert der gleichphasigen Schwingung $\omega_{gl}^T = 3,217(57)$,
    der mithilfe von (\ref{D}) und (\ref{J}) berechnet wurde,
    liegt im Konfidenzintervall beider Messreihen. \\
    
    Im Graphen \ref{fig:plotK} werden die aus Messreihe 2 und 3 bestimmten Kopplungskonstanten gezeigt.

    Mit diesen Graphen können durch durch gefitteten Funktionen (\ref{Kkappa}) bzw. (\ref{Okappa}) die beiden
    Federkonstanten bestimmt werden.
    \begin{table}[H]
        \centering
        \begin{tabular}{c c c}
            Federkonstante & \textcolor{blue}{Feder 1} & \textcolor{red}{Feder 2} \\ \hline
            mit $\omega_{geg}$ & $4.00 \si{\newton\per\metre}$ & $5.90\si{\newton\per\metre}$\\
            mit $K$ & $4.01 \si{\newton\per\metre}$ & $5.65\si{\newton\per\metre}$\\
            
        \end{tabular}
        \caption{Berechnete Federkonstanten}
    \end{table}


    \subsection{Zusammenfassung}
    Insgesamt sind die Ergebnisse in sich stimmig und mit den Theoriewerten vereinbar. Wie zu erwarten, gab
    je nach Komplexität des fittens größer oder kleinere Fehler. Durch die automatisierte Datenaufnahme
    wahr zwar beim zweiten Versuch die Durchführung einfacher, aber die Auswertung durchaus schwerer.
    Es ist erstaunlich, mit was für komplizierten
    Versuchsaufbauten so einfache Eigenschaften wie die Federkonstante gemessen werden können.




    \section{Anhang}
    \subsection{Fehlerberechung 1}
    
    \paragraph{Unsicherheit $l_r$:}
    $l_r$ wurde mit einem Metalllineal gemessen werden. Deshalb kann mithilfe von \cite[Gleichung (40)]{ABW}
    und \cite[Tabelle 5]{ABW} die Unsicherheit des Lineals (Länge L=1m)
    \begin{align}
        u(L) = 0.5\si{\milli\metre}
    \end{align}
    und mit \cite[Tabelle 1]{ABW} die Ablesegenauigkeit
    \begin{align}
        u_a = \frac{0.5\si{\milli\metre}}{2\sqrt{3}} = 0.15\si{\milli\metre}
    \end{align}
    berechnet werden. Das Metallineal hat jede 0.5mm ein Ablesestrich.
    Die Unsicherheiten werden zur Gesamtunsicherheit
    \begin{align}
        u_l = \sqrt{u(L)^2 + u_a^2 } = 0.52\si{\milli\metre}
    \end{align} 
    zusammengefasst.
    \paragraph{Unsicherheit $\tau$:}
    Die Unsicherheiten wurden mithilfe von der Formel in der Aufgabenstellung \cite[Aufgabe 8]{pen} berechnet.

    \paragraph{Unsicherheiten von $g_1$ und $g_2$:}
    Mit der Formel \ref{gformel} kann die allgemeinen Formel für Fehlerfortpflanzung
     \cite[Formel (20)]{ABW} für diesen Fall angepasst werden:
     \begin{align}
         u(\bar{g}) &= \sqrt{\left(\frac{\partial g}{\partial l_r}\right)^2 u(l_r)^2 +
         \left(\frac{\partial g}{\partial \theta}\right)^2 u(\theta)^2} \nonumber \\
         &= \sqrt{\frac{16\pi^4}{\theta^2} u(l_r)^2 + \frac{64\pi^4l_r^2}{\theta^3} u(\theta)^2}
     \end{align}
    \paragraph{Gewichteter Mittelwert:}
    Der Gewichtete Mittelwert und seine Unsicherheit wurde mithilfe von
    Formeln (29) bis (32) im ABW Skript\cite{ABW} berechnet. da $u_{int}$ mit $0.023\si{\metre\per\square\second}$
    größer als $u_{ext} = 0.00019\si{\metre\per\square\second}$ ist, wird $u_{int}$ als Ungenauigkeit angenommen.

    \subsection{Messwerte gekoppelte Pendel} \label{kmess}

    \begin{table}[H]
        \label{mess1}
        \centering
        \begin{tabular}{c c c c c}
            Feder & Abstand $r$ in \si{\centi\metre} & $\omega_{geg}$ in \si{\radian\per\second} &
            $\omega_{gl}$ in \si{\radian\per\second} & $K$ \\ \hline
            
            \input{KdataTabel.txt}
        \end{tabular}
        \caption{Übersicht der Daten der 1. und 2. Messreihe}
    \end{table}

    \begin{table}[H]
        \label{mess2}
        \centering
        \begin{tabular}{c c c c c c c}
            Feder & Abstand $r$ in \si{\centi\metre} & $\omega_{M}$ in \si{\radian\per\second} & $\omega_{S}$ in \si{\radian\per\second} & $K$ \\ \hline
            
            \input{SChwdataTabel.txt}
        \end{tabular}
        \caption{Übersicht der Daten der 3. Messreihe}
    \end{table}


    \bibliographystyle{plain}
    \bibliography{literature}


\end{document}