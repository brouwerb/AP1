\documentclass[11pt, a4paper]{article}

\usepackage{amsmath}
\usepackage{graphicx}
\usepackage[ngerman]{babel}
\addto\captionsngerman{%
 \renewcommand{\abstractname}{Abstract}}

\title{Versuch 2: Pendel}
\author{Jascha Fricker, Benedict Brouwer}

\begin{document}
    \maketitle

    

    \begin{abstract}
        In diesen beiden Versuchsaufbauten werden verschiedene Pendel untersucht. Zum einen wird mit
        einem Reversionspendel die Erdbeschleunigung $g$ gemessen, zum anderen werden zwei mit einer Feder
        gekoppelte Pendel untersucht und mit den Messwerten u.a. die Federkonstante berechnet.
    \end{abstract}

    \tableofcontents

    \section{Reversionspendel}

    \subsection{Experimenteller Aufbau und Theorie}

    Ein Reversionspendel hat zwei Aufhängepunkte und zwei Massen, die alle auf einer Geraden liegen.
    Dabei kann ein Masse verschoben werden.
    Es gibt zwei Positionen des verschiebbaren Gewichts, an dem die Periode der Schwingung an beiden
    Aufhängepunkten gleich ist. Für ein Reverionspende in dieser Konfiguration ist die Periode
    \begin{align}
        \tau^2 = 4\pi^2 \cdot \frac{l_r}{g}
    \end{align}
    (Herleitung siehe ?) Daraus ist ersichtlich, dass, wenn dieser Fall eintritt, die Periodendauer unabängig von der Masse und des
    Trägheitsmoments ist. So kann mit der Periodendauer und dem Abstand der beiden Aufhängungspunkte $l_r$
    die Erdbeschleunigung ausgerechnet werden.
    Um diese besonderen Positionen der zweite Masse zu finden, wird im Experiment die Periodendauer
    mit der Masse an verschieden Positionen von beiden Aufhängungen gemessen,
    um dann mithilfe des Schnittpunkts der Ausgleichgeraden die gewünschten
    Punkte bestimmt.



    \subsection{Ergebnisse}

    In den Graphen ? und ? können die beiden Schnittpunkte gesehen werden. Da der Fehler der Zeitmessung
    nur etwa $2ms$ beträgt, sind die Fehlerbalken sehr nah am Graphen. Der Abstand der beiden
    Aufhängungspunkte $l_r$ beträgt $800,00(25)cm$

    \begin{table}
        \centering
        \begin{tabular}{c|c|c}
            Schnittpunkt & 1 & 2 \\ \hline
            Periodendauer & & \\ \hline
            Erdbeschleunigung & &
        \end{tabular}
        \caption{Ergebisse}
        \label{ergrev}
    \end{table}

    \section{Gekoppelte Pendel}

    \subsection{Experimenteller Aufbau und Theorie}



    \subsection{Ergebnisse}


\end{document}