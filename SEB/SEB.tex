\documentclass[11pt, a4paper]{article}

\usepackage{amsmath}
\usepackage{amsfonts} %Matheschriften
\usepackage{amssymb} %Mathesymbole
%\usepackage{mathptmx} % Einstellung für Schriften und Sonderzeichen in mathematischen Umgebungen
                        % ändert SChriftfont
\usepackage{wasysym} % Stellt diverse Sonderzeichen bereit
\usepackage{siunitx}
\usepackage{float}
\usepackage{microtype}
\usepackage{graphicx}
\usepackage{hyperref}
\usepackage{xcolor}
\usepackage[section]{placeins}


\usepackage[ngerman]{babel}
\addto\captionsngerman{%
 \renewcommand{\abstractname}{Abstract}}

\title{Versuch 3: Pohlsches Rad}
\author{Jascha Fricker, Benedict Brouwer}

\begin{document}
    \maketitle

    

    \begin{abstract}
        In diesen Versuch wird ein gedämpfter harmonischer Oszillator untersucht. Harmonische Oszillatoren
        sind ein wichtiges Modell in der Physik, da sie sehr viele systeme in der realen Welt beschreiben können.
        Ein berühmtes Thema sind z. B. Resonanzkatastrophen, bei dehnen große Bauwerke in iherer Eingenschwingung
        angeregt werden. Genau solche Schwingungen eines getriebenen Harmonischen Oszillators werden auch in diesem
        Versuch untersucht.
    \end{abstract}

    \tableofcontents

    \newpage

    \section{Theorie}

    


    \bibliographystyle{plain}
    \bibliography{literature}

\end{document}