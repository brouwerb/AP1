\documentclass[11pt, a4paper]{article}

\usepackage{amsmath}
\usepackage{amsfonts} %Matheschriften
\usepackage{amssymb} %Mathesymbole
%\usepackage{mathptmx} % Einstellung für Schriften und Sonderzeichen in mathematischen Umgebungen
                        % ändert SChriftfont
\usepackage{wasysym} % Stellt diverse Sonderzeichen bereit
\usepackage{siunitx}
\usepackage{float}
\usepackage{microtype}
\usepackage{graphicx}
\usepackage{hyperref}
\usepackage{xcolor}
\usepackage[section]{placeins}


\usepackage[ngerman]{babel}
\addto\captionsngerman{%
 \renewcommand{\abstractname}{Abstract}}

\title{Versuch 3: Pohlsches Rad}
\author{Jascha Fricker, Benedict Brouwer}

\begin{document}
    \maketitle

    

    \begin{abstract}
        In diesen Versuch wird ein gedämpfter harmonischer Oszillator untersucht. Harmonische Oszillatoren
        sind ein wichtiges Modell in der Physik, da sie sehr viele systeme in der realen Welt beschreiben können.
        Ein berühmtes Thema sind z. B. Resonanzkatastrophen, bei dehnen große Bauwerke in iherer Eingenschwingung
        angeregt werden. Genau solche Schwingungen eines getriebenen Harmonischen Oszillators werden auch in diesem
        Versuch untersucht.
    \end{abstract}

    \tableofcontents

    \newpage

    \section{Theorie}

    Ein gedämpfter harmonischer Oszillator hat die Bewegungsgleichung
    \begin{align}
        \phi(t) = C \cdot exp(-\lambda t) \cdot cos(\omega_d - \beta),
    \end{align}
    mit Kreisfrequenz $\omega_d = \sqrt{\frac{k}{\Theta} - \lambda^2}$ (Drehmoment $\Theta$ und Federkonstante $k$)
    und $\beta$ als Phasenverschiebung (Herleitung siehe Aufgabenblatt \cite{POR}). Bei der in diesem Experiement benutzten Wirbelstrombremse ist die
    Dämpfungskonstante proportional zum Quadrat des Stroms 
    \begin{align}
        \lambda \propto I^2.
    \end{align}
    Die Eigenfrequenz lässt sich auch mithilfe der Schwingungsperiode $T$ berechnen
    \begin{align}
        \omega_d = \frac{2\pi}{T}
    \end{align}
    Bei einem gezwungenen Oszillator mit Drehmoment $M_0 sin(\omega t)$ kommt noch die partikuläre Lösung zur Bewegungsgleichung
    \begin{align}
        \phi(t) &= A(\omega) \cdot sin(\omega t - \Phi) + C \cdot exp(-\lambda t) \cdot cos(\omega_d - \beta) \\
        \text{mit Amplitude}  \ \ A(\omega) &= \frac{M_o}{\Phi \sqrt{}}
    \end{align}


    \bibliographystyle{plain}
    \bibliography{literature}

\end{document}