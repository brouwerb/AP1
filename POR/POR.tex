\documentclass[11pt, a4paper]{article}

\usepackage{amsmath}
\usepackage{amsfonts} %Matheschriften
\usepackage{amssymb} %Mathesymbole
%\usepackage{mathptmx} % Einstellung für Schriften und Sonderzeichen in mathematischen Umgebungen
                        % ändert SChriftfont
\usepackage{wasysym} % Stellt diverse Sonderzeichen bereit
\usepackage{siunitx}
\usepackage{float}
\usepackage{microtype}
\usepackage{graphicx}
\usepackage{hyperref}
\usepackage{xcolor}
\usepackage[section]{placeins}


\usepackage[ngerman]{babel}
\addto\captionsngerman{%
 \renewcommand{\abstractname}{Einleitung}}

\title{Versuch 4: Pohlsches Rad}
\author{Jascha Fricker, Benedict Brouwer}

\begin{document}
    \maketitle

    

    \begin{abstract}
        In diesen Versuch wird ein gedämpfter harmonischer Oszillator untersucht. Harmonische Oszillatoren
        sind ein wichtiges Modell in der Physik, da sie sehr viele systeme in der realen Welt beschreiben können.
        Ein berühmtes Thema sind z. B. Resonanzkatastrophen, bei dehnen große Bauwerke in iherer Eingenschwingung
        angeregt werden. Genau solche Schwingungen eines getriebenen Harmonischen Oszillators werden auch in diesem
        Versuch untersucht.
    \end{abstract}

    \tableofcontents

    \newpage

    \section{Stromsbhängigkeit der Dämpfungskonstanten}
    \subsection{Theorie}
    Je nachdem wie stark ein harmonischer Oszillator gedämft wird, ändert sich seine Schwingfrequenz.
    Ein gedämpfter harmonischer Oszillator hat die Bewegungsgleichung
    \begin{align}
        \phi(t) = \phi_0 \cdot exp(-\lambda t) \cdot cos(\omega_d - \beta), \label{theo5}
    \end{align}
    mit Kreisfrequenz $\omega_d = \sqrt{\frac{k}{\Theta} - \lambda^2}$ (Drehmoment $\Theta$ und Federkonstante $k$)
    und $\beta$ als Phasenverschiebun. Diese kann aus der Differtialgleichung der Kräfte 
    hergeleitet werden (siehe Aufgabenblatt \cite[(5)]{POR}).
    Bei der in diesem Experiement benutzten Wirbelstrombremse ist die
    Dämpfungskonstante $\lambda$ proportional zum Quadrat des Stroms $I$ durch die Bremse.
    \begin{align}
        \lambda \propto I^2.
    \end{align}

    \subsection{Experimenteller Aufbau}
    Die Stromstärke der Wirbelstrombremse wurde mit einem Labornetzteil eingestellt und
    mit einem VC130 Multieter gemessen. Nachdem das Phlsche Rad fast maximal ausgelenkt worden war, 
    wurde der Winkel des Pohlschen Rades abhängig von der Zeit
    mit einem Hall Sensor gemessen und direkt digital aufgenommen.

    \subsection{Auswertung}
    Um die Eingenfrequenz und Dämpfungskonsante abhängig von der Stromstärke zu bestimmen,
    wurde auf jede Messreihe einzeln mit
    der Funkion \textsf{optimize.curve\_fit} der Pythonbibliothek scipy die Theoriekurve \ref{theo5}
    gefittet und auch die Unsicherheit ausgerechnet. Mit diesen 2x15 Datenpunkten können jetzt die Zusammenhänge
    zwischen $\omega_d, \lambda$ und $I$ untersucht werden.


    Die Eigenfrequenz lässt sich auch mithilfe der Schwingungsperiode $T$ berechnen
    \begin{align}
        \omega_d = \frac{2\pi}{T}
    \end{align}


    Bei einem gezwungenen Oszillator mit Drehmoment $M_0 sin(\omega t)$ kommt noch die partikuläre Lösung zur Bewegungsgleichung
    \begin{align}
        \phi(t) &= A(\omega) \cdot sin(\omega t - \Phi) + C \cdot exp(-\lambda t) \cdot cos(\omega_d - \beta) \\
        \text{mit Amplitude}  \ \ A(\omega) &= \frac{M_o}{\Phi \sqrt{}}
    \end{align}


    \bibliographystyle{plain}
    \bibliography{literature}

\end{document}