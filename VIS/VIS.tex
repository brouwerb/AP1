\documentclass[11pt, a4paper]{article}

\usepackage{amsmath}
\usepackage{amsfonts} %Matheschriften
\usepackage{amssymb} %Mathesymbole
%\usepackage{mathptmx} % Einstellung für Schriften und Sonderzeichen in mathematischen Umgebungen
                        % ändert SChriftfont
\usepackage{wasysym} % Stellt diverse Sonderzeichen bereit
\usepackage{siunitx}
\usepackage{float}
\usepackage{microtype}
\usepackage{graphicx}
\usepackage{hyperref}
\usepackage{xcolor}
\usepackage[section]{placeins}
% allows for temporary adjustment of side margins
\usepackage{changepage}


\usepackage[ngerman]{babel}
\addto\captionsngerman{%
 \renewcommand{\abstractname}{Einleitung}}

\title{Versuch 4: Pohlsches Rad}
\author{Jascha Fricker, Benedict Brouwer}

\begin{document}
    \maketitle

    

    \begin{abstract}
        In diesen Versuch wird die Viskosität von verschiedenen Fluiden untersucht. Mit der Viskosität
        und der Reynoldzahl kann ausgerechnet werden, ob eine Strömung turbulent oder laminar ist.
        Dies ist in vielen Bereichen wichtig, da so z. B. ermittelt werden kann, ob ein Flugzeug fliegt oder
        aus dem Himmel fällt.
    \end{abstract}

    \tableofcontents

    \newpage

    \section{Kugelfallviskosimeter}

    \subsection{Theorie}

    Bei einem Kugelfallviskosimeter wird die dynamische Viskosität $\eta$ des Fluides mit 
    der Fallgeschwindigkeit $v$ einer Kugel
    mit gegebenem Radius $r$ und Masse $m$ bestimmt.
    Bei annahme eines unendlich ausgedehnten Mediums, gilt
    \begin{align}
        \eta = \frac{2r^2g}{9v}(\rho_K - \rho_F)
    \end{align}
    mit Dichte der Kugel $rho_K$ und Dichte der Flüssigkeit $rho_F$.
    Wird hingegen ein unendlich langer Zylinder mit Radius $R$ betrachtet, gilt
    \begin{align}
        \eta = \frac{2R^2g}{9v(1+2,4\frac{r}{R})}(\rho_K - \rho_F)
        \label{eq:cylinder}
    \end{align}
    Um daraus die Reynoldszahl zu berechnen, gilt
    \begin{align}
        \Re = \frac{2r \rho_F v}{\eta}
    \end{align}

    \subsection{Experimenteller Aufbau}
    In diesem Experiment wurde 17 mal die Kugel durch die Flüssigkeit fallen gelassen und
    jedes mal wurde über einen bestimmten Abstand $s$ die Fallzeit $t$ gemessen.
    Anschließend wurde noch die Dichte der Flüssigkeit $\rho_F$ mit einem Aerometer gemessen.
    
    \subsection{Ergebnisse}

    Die Fehler, die betrachtet wurden, sind in der Tabelle ? aufgeführt.



    \section{Anhang}
    \subsection{Kugelfallviskosimeter}

    \subsubsection{Berechnung der dynamischen Viskosität mit Fehler}
    
    \begin{table}
       \begin{adjustwidth}{-5cm}{-5cm}
            \centering
            \begin{tabular}{c c c}
                Messgröße & Fehler & Begründung \\ \hline
                Gewicht $m$ & $u_g = 0,0005 \si{\gram}$ & Skalierung \\
                Radius Kugel $r$ & $u_r = 0,05 \si{\milli\meter}$ & ABW Skript \cite[Tabelle 6]{ABW} \\
                Radius Zylinder $R$ & $u_R = 0,06 \si{\milli\meter}$ & ABW Skript \cite[Tabelle 6]{ABW} \\
                Zeit $t$ & $u_t = \sqrt{2} 0,3\si{\second} = 0,425 \si{\second}$ & Reaktionszeit x2\\
                Strecke $s$ & $u_s = \frac{1}{2\sqrt{6}} \si{\milli\metre} = 0.21 \si{\milli\metre}$ & Schrittweite $1 \si{\milli\metre}$ \\
                Dichte Flüssigkeit $\rho_F$ & $u_{\rho_F} = \frac{0,01}{2\sqrt{6}} \si{\gram \per \milli\litre} = 0,0021  \si{\gram\per\milli\litre}$ & Schrittweite $0,01 \si{\kilogram\per\cubic\metre}$ \\

            \end{tabular}
        \end{adjustwidth}
    \end{table}
    
    Als erstes wurde der Mittelwert der Fallzeiten berechnet.
    \begin{align}
        \bar{t} &= \frac{1}{n} \sum_{i=1}^{n} t_i = 3,94(15) \si{\second}
    \end{align}
    Der Fehler der Fallzeiten wurde mithilfe der Student-t-Verteilung (n=10) berechnet.
    \begin{align}
        u_{\bar{t}} &= \frac{{\it t}}{\sqrt{n}} u_{t_i} =  0,144 \si{\second}
    \end{align}
    Auch vom Kugeldurchmesser kann erstmal der Mittelwert
    \begin{align}
        \bar{r} &= \frac{1}{n} \sum_{i=1}^{n} r_i = 3,998(17) \si{\milli\metre}
    \end{align}
    und dann der Fehler mit Student-t (n=10)
    \begin{align}
        u_{\bar{n}} &= \frac{{\it t}}{\sqrt{n}} u_{r_i} = 0.017 \si{\milli\metre}
    \end{align}
    berechnet werden

    Jetzt kann $\eta$ mit Formel \cite[(11)]{VIS}
    \begin{align}
        \eta &= \frac{2r^2g}{9v}(\rho_K - \rho_F) \\
        &= \frac{2r^2g}{9\frac{s}{\bar{t}}}(\frac{m}{\frac{4}{3} \pi r^3} - \rho_F)
        &= 
    \end{align}
    und der Fehler mit der Gausschen Fehlerfortpflanzung \cite{ABW}
    \begin{multline}
        \sqrt{\sum_{i=0}^{4} 18.9988739756689 \left(\frac{\left(- {x}_{4} + \frac{0.75 {x}_{2}}{\pi {x}_{0}^{3}}\right) \delta_{0 i} {x}_{0} {x}_{3}}{{x}_{1}} - \frac{0.5 \left(- {x}_{4} + \frac{0.75 {x}_{2}}{\pi {x}_{0}^{3}}\right) \delta_{1 i} {x}_{0}^{2} {x}_{3}}{{x}_{1}^{2}} + \frac{0.5 \left(- {x}_{4} + \frac{0.75 {x}_{2}}{\pi {x}_{0}^{3}}\right) \delta_{3 i} {x}_{0}^{2}}{{x}_{1}} + \frac{0.5 \left(- \frac{2.25 \delta_{0 i} {x}_{2}}{\pi {x}_{0}^{4}} + \frac{0.75 \delta_{2 i}}{\pi {x}_{0}^{3}} - \delta_{4 i}\right) {x}_{0}^{2} {x}_{3}}{{x}_{1}}\right)^{2} {ui}_{i}^{2}}
    \end{multline}


    \bibliographystyle{plain}
    \bibliography{literature}

\end{document}